\documentclass[a4paper,12pt]{article}
\usepackage[top = 2.5cm, bottom = 2.5cm, left = 2.5cm, right = 2.5cm]{geometry}
\usepackage[T1]{fontenc}
\usepackage[utf8]{inputenc}

% The following two packages - multirow and booktabs - are needed to create nice looking tables.
\usepackage{multirow} % Multirow is for tables with multiple rows within one cell.
\usepackage{booktabs} % For even nicer tables.

\usepackage{graphicx}

% \usepackage{setspace}
% \setlength{\parindent}{0in}

\usepackage{float}

\usepackage{fancyhdr}
\usepackage{fancyvrb}
\pagestyle{fancy}
\usepackage{indentfirst}
\setlength{\parindent}{2em}
\renewcommand{\baselinestretch}{1.5}

\fancyhf{}

\lhead{\footnotesize Trabalho Aula 08}

\rhead{\footnotesize }

\cfoot{\footnotesize \thepage}

\begin{document}


\thispagestyle{empty}

\begin{tabular}{p{15.5cm}}

{\large \bf Teoria da Computação} \\
UNESP - Univerisade Estadual Paulista - Presidente Prudente\\ SEM1 2020  \\ Celso\\
\hline
\\
\end{tabular}

\vspace*{0.3cm}

\begin{center} % Everything within the center environment is centered.
	{\Large \bf Trab. Aula 08: Computador Digital (Von Neumann)} % <---- Don't forget to put in the right number
	\vspace{2mm}

        % YOUR NAMES GO HERE
	{\bf Gilmar Francisco de Oliveira Santos} % <---- Fill in your names here!

\end{center}

\vspace{0.4cm}

\section{Máquina de Turing Multifita para a Arquitetura}
\begin{verbatim}
    A arquitetura construída consta no arquivo em anexo "trab_08.jff"
    Ou no link: "http://github.com/gilmarfrancisco828/trab08.jff"
\end{verbatim}
\section{Casos de Teste}

\subsection*{Caso 4: NOT}
\begin{Verbatim}[numbers=left,xleftmargin=8mm]
    Carrega #1000*1101#0*1011000#1*011#10*1101000# na fita 1
    Inicia a fita 2 com 0
    Busca na memória(Fita 1) o endereço 0
    Lê a operação de LOAD do endereço 1000 no reg X(fita 3)
    Busca na memória o endereço 1000
    Copia o conteúdo do endereço 1000 no reg X(fita 3).
    Incrementa PC para 1
    Lê o conteúdo de PC, busca na memória a instrução no endereço 1
    Lê a operação NOT
    Inverte os bits do reg X
    Incremente PC para 10
    Lê o conteúdo de PC, busca na memória a instrução no endereço 10
    Lê a operação 110(STORE) do conteúdo do reg X para o endereço 1000
    Copia o conteúdo de X para o início da memória, salvo com o endereço 1000
    Incrementa PC para 11
    Busca a instrução no endereço 11, não encontra -> Halt
    Finaliza execução
\end{Verbatim}
\section{Solução Adotada}

\section*{Ciclo de Funcionamento}
A execução é inicializada pelo carregamento da entrada na Fita 1, que será a memória de trabalho, então o valor $0$ é pré-carregado na Fita 2, a qual será representado o PC, inicia-se então o ciclo de busca e execução de instruções, cada instrução encontrada, representada pelos 3 primeiros bits do conteúdo do endereço alvo, é executada de acordo com o que se segue abaixo, para então ocorrer uma operação de incremento do PC, usando da própria unidade de Adição, com exceção da operação de JUMP, que carrega diretamente o valor no PC, e o ciclo de busca é reiniciado.

\subsection*{Gerenciamento de Memória}
A busca é realizada na memória, Fita 1, de maneira linear, estando os endereços presentes na memória disposto de maneira aleatória, para a busca de endereços de intruções o valor utilizado é o do PC, enquanto que para endereços de variáveis, o endereço é alocado temporariamente na fita 5, para então a busca linear ser iniciada.

As escritas na memória são feitas no início, dadas as complexidades envolvidas ao sobrescrever posições de memória no meio, com isso a busca de endereços sempre irá utilizar o primeiro endereço encontrado, começando sempre da direita, de modo a garantir que novos valores para as posições de memória sejam possíveis.

Endereços não Encontrados fazem a MT travar, porém caso seja um endereço fornecido pelo PC e não for encontrado, a máquina finaliza a execução, vai para o estado de Halt.

\section*{STORE}
Rebobina a memória, ou seja, volta para o início, e escreve da esquerda para direita, primeiro o conteúdo do registrador X, adiciona o carácter de separação "*", escreve então o endereço alvo e adiciona o carácter "\#", finalizando a adição ou sobrescrita da nova posição de memória.

Então chama a operação de Incrementar o PC e busca a próxima instrução.

\section*{LOAD}
Busca na memória o endereço alvo, ao encontrar, copia seu conteúdo diretamente para o registrador X.

Então chama a operação de Incrementar o PC e busca a próxima instrução.

\section*{JUMP}
A operação de JUMP copia o endereço presente na memória diretamente para a Fita 2(PC) e retorna para o estado de Busca de Instrução.

\section*{SWAP}
A operação de SWAP troca os conteúdos dos registradores X e Y, usando a fita 5 como intermediária, primeiramente copia o conteúdo de X para a fita 5, depois copia o valor de Y para X, para então copiar o valor da fita 5 para Y.

Então chama a operação de Incrementar o PC e busca a próxima instrução.

\section*{NOT}
A operação lógica NOT é realizada diretamente no registrador X, a MT itera sobre todos os bits e inverte seus valores diretamente na própria fita.

Então chama a operação de Incrementar o PC e busca a próxima instrução.

\section*{AND}
A operação de AND lógico é realizada da esquerda para direita, aplicando a tabela verdade do operador lógico AND simultâneamente aos registradores X e Y, quando um \textbf{branco} é encontrado em um dos registradores, zero é escrito na posição correspondente.

Então chama a operação de Incrementar o PC e busca a próxima instrução.
% COLOCAR QUAL FITA QUE É SALVO
\section*{ADD}

Para a operação de AND, a Fita 5 é majoritariamente utilizada, para isso é feito a cópia dos valores contidos no registrador X e Y para a Fita 5, onde a operação a ser realizada é no seguinte formato:
\begin{verbatim}
    #b_x#b_y#d
\end{verbatim}
Onde $b_x$ e $b_y$ representam respectivamente os valores copiados dos registradores X e Y em binário, e $d$ representa um código para a escolha de qual será o destino do resultado, sendo os valores possíveis: F(resultado vai ser copiado no Reg. x), .

Então chama a operação de Incrementar o PC e busca a próxima instrução.

\section*{SUB}

A operação de subtração é feita reutilizando-se o a estrutura para Adição, porém a estratégia é utilizar a \textbf{soma binária com complemento de 1}, sendo assim o valor do Reg. Y é copiado para a Fita 5 com seus valores de Bits invertidos e então somado com o valor do Reg. X, o resultado é intermediário, então descarta-se o primeiro bit(carry) do resultado, dado que só trabalhamos com números positivos, esse resultado é então somado com $1$, para só então seu resultado ser salvo no Reg. X. Caso os valores de X e Y sejam iguais, a MT escreve diretamente zero no Reg X.

Então chama a operação de Incrementar o PC e busca a próxima instrução.
\end{document}
